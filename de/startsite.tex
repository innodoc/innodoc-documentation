\MSection{Willkommen}

\begin{MSectionStart}
\MDeclareSiteUXID{BASE_FIRSTPAGE}
\MGlobalStart % Diese Seite erscheint beim erstmaligen Aufrufen der Onlinemodule
\MPullSite % Beim Aufruf dieser Seite werden die Benutzerdaten aus der Datenbank geholt (falls aktiv)

% Der Konverter versteht die meisten LaTeX-Kommandos, Umgebungen und Stile
Dies ist das Basismodul des \texttt{tex2x}-Konverters des VE\&MINT-Projekts.

% HTML-Stücke können dennoch im LaTeX direkt eingegeben werden mit der html-Umgebung
\begin{html}
<div style="background-color:rgb(220,220,220);padding:10px">
<h1 class="start" >Basis-Beispielmodul</h1>
<p>
Willkommen zum Beispielmodul.
Kursdaten werden im Browser automatisch gespeichert, eine explizite Anmeldung zum Modul ist nicht erforderlich.
<noscript>
<b>JavaScript ist auf diesem Browser nicht aktiviert, ohne Javascript kann der Kurs nicht ausgeführt werden!</b>
</noscript>
<br />
<center><button class="stdbutton" type="button" onclick="opensite(':directmaterial:1.1.1/modstart.html');">Inhaltsbereich öffnen</button></center>
<br />
</p>
</div>
\end{html}

% In der Optionendatei angegebene Flags können mit if abgefragt werden, vorher sollte mit ifttm aber immer geprüft werden,
% ob tatsächlich eine Onlinekonvertierung (ifttm=true) oder eine PDF-Umsetzung stattfindet
\ifttm%
\ifnum\value{MCFlagdorelease}=0% Falls dorelease=0 in Optionen oder Kommandozeile: Füge eine Analyse-Tests-Box ein
\begin{html}
<div style="background-color:rgb(220,190,55);padding:30px">
<h2 class="start" >TESTVERSION - NICHT ZUR VERÖFFENTLICHUNG GEDACHT</h2>
<p>
<noscript>
<b>JavaScript ist auf diesem Browser nicht aktiviert, ohne Javascript kann der Kurs nicht ausgeführt werden!</b>
</noscript>
<br />
<center>
<div id="ANALYTICSECTION">
Hier werden Analysedaten eingetragen.
</div>
</center>
<br />
</p>
</div>
\end{html}
\fi%
\fi%

\ \\ \ \\
% Diverse Optionen zur Modulerstellung können über Makrobefehle abgerufen werden
\begin{tabular}{lllll}
Kursversion: & \MSignatureMain (\MSignatureVersion) & \ \ &
Erstellung: & \MSignatureDate\\
Lokale Version: & \MSignatureLocalization & \ \ & 
Kursvariante: & \MSignatureVariant\\
\end{tabular}

\ \\
% Ein paar technsiche Spielereien um auf den Browsertyp hinzuweisen
\begin{html}
<p>
<h3  class="start">Welche Browser kann man verwenden?</h3>
Die folgenden Browser k&ouml;nnen verwendet werden: Firefox, Internet Explorer, Chrome, Safari, Opera.<br />
Andere Browser haben zum Teil Probleme, unsere Modulseiten richtig anzuzeigen.
<br />
Wir empfehlen, nur die neusten Versionen dieser Browser mit den aktuellen Updates einzusetzen,
insbesondere kann der Kurs nicht mit veralteten Browsern (Internet Explorer 8 oder früher) bearbeitet werden.
<br />
<br />
<script type="text/javascript">
document.write("Ihr Browsertyp: " + navigator.appName + ", Browserkennung: " + navigator.userAgent);
</script>
<br />
Unsere Seiten verwenden aktive Inhalte, insbesondere <a class="start" href="http://www.mathjax.org" target="_new">MathJax</a>. Dazu muss JavaScript im Browser aktiviert sein.
<br />
<a href="http://www.mathjax.org"><img title="Powered by MathJax" src="images/mj_logo.png" border="0" alt="Powered by MathJax" /></a><br />
<br />
<script type="text/javascript">
document.write("JavaScript ist auf diesem Browser aktiviert.");
</script>
</p>
<br/>
\end{html}

\end{MSectionStart}

\MSubsection{Informationen und Impressum}

\begin{MXContent}{Haftungsauschluss}{Haftungsauschluss}{STD}
\MLabel{L_HAS}
\MDeclareSiteUXID{BASE_LEGAL}

% Ein generischer Haftungsauschluss für Inhalte gemäß CC BY-SA
\MSubsubsubsectionx{Funktionalität und Gewährleistung}
Die im Rahmen dieses Kurses eingesetzten Materialien und Softwarebausteine unterliegen einer freien Lizenz, aber erheben keinen Anspruch auf inhaltliche oder technische Fehlerfreiheit. Die Inhalte wurden
sorgsam überprüft, dennoch kann insbesondere während der Anlaufphase des Kursbetriebs keinerlei Gewähr auf Richtigkeit oder technische Fehlerfreiheit der Inhalte und Software sowie auf
Verfügbarkeit der Betriebsserver übernommen werden. Dies bestrifft sowohl die clientseitigen Softwarekomponenten (insbesondere HTML5 und JS-Code) sowie die über unsere Server bezogenen Inhalte.
Für auftretende clientseitige Probleme wie z.B. fehlerhaftes Browserverhalten oder optisch verfälschende Wiedergabe unserer Inhalte können wir aufgrund der Vielzahl der Browser und der häufigen
Sicherheitsupdates für die gängigen Betriebssysteme ebenfalls keine Gewährleistung geben.

\MSubsubsubsectionx{Inhalt und Informationsübermittlung}
Der Diensteanbieter ist für eigene Inhalte auf diesen Seiten nach den allgemeinen Gesetzen verantwortlich,
jedoch nicht verpflichtet, übermittelte oder gespeicherte fremde Informationen zu überwachen oder nach Umständen zu forschen,
die auf eine rechtswidrige Tätigkeit hinweisen. Verpflichtungen zur Entfernung oder Sperrung der Nutzung von Informationen nach den allgemeinen Gesetzen bleiben hiervon unberührt.
Eine diesbezügliche Haftung ist jedoch erst ab dem Zeitpunkt der Kenntnis einer konkreten Rechtsverletzung möglich. Bei Bekanntwerden von entsprechenden Rechtsverletzungen werden wir diese Inhalte umgehend entfernen.

\MSubsubsubsectionx{Haftung für Links}
Unser Angebot enthält Links zu externen Webseiten Dritter, auf deren Inhalte wir keinen Einfluss haben. Deshalb können wir für diese fremden Inhalte auch keine Gewähr übernehmen.
Für die Inhalte der verlinkten Seiten ist stets der jeweilige Anbieter oder Betreiber der Seiten verantwortlich. Die verlinkten Seiten wurden zum Zeitpunkt der Verlinkung
auf mögliche Rechtsverstöße überprüft. Rechtswidrige Inhalte waren zum Zeitpunkt der Verlinkung nicht erkennbar. Eine permanente inhaltliche Kontrolle der verlinkten Seiten ist jedoch ohne
konkrete Anhaltspunkte einer Rechtsverletzung nicht zumutbar. Bei Bekanntwerden von Rechtsverletzungen werden wir derartige Links umgehend entfernen.

\MSubsubsubsectionx{Urheberrecht}
Die durch das VE\&MINT-Projekt erstellten Inhalte sowie die eingesetzte Softare auf diesen Seiten unterliegen dem deutschen Urheberrecht.
Die Vervielfältigung, Bearbeitung, Verbreitung ist gemäß der \MExtLink{http://de.wikipedia.org/wiki/Creative_Commons}{Creative Commons License} CC BY-SA in der Version 3.0 erlaubt.
Von Dritten eingebrachte oder übernommene Inhalte stehen ebenfalls unter dieser Lizenz. Urheber- und Autorenrechte bleiben davon unberührt.
Sollten Sie auf eine Urheberrechtsverletzung aufmerksam werden, bitten wir um einen entsprechenden Hinweis. Bei Bekanntwerden von Rechtsverletzungen werden wir derartige Inhalte umgehend entfernen. 
\end{MXContent}
