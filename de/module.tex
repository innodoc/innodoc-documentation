% Damit die Moduldatei auch direkt mit pdflatex umgesetzt werden kann, müssen die Makros eingebunden werden
\input{mintmod.tex}


\MPragma{MathSkip} % Pragmas legen Konvertierungsoptionen nur für diese Datei fest. Diese Pragma fügt Abstände vor displaystyle-Matheumgebungen ein

\begin{document}

%\Mtableofcontents % erscheint nur bei PDF-Konvertierung, im HTML gibt es eine separate Seitennavigation

\MSection{Beispielmodul}

% Jede MSection besteht aus einer MSectionStart-Umgebung, dann MSubsections mit darin enthaltenen MXContent-Umgebungen.

\begin{MSectionStart}
% Jede Seite und jedes Frageelement haben eine UXID (unique extended ID), diese ist ein
% innerhalb des Kurses eindeutiger Textstring der die Seite unabhängig von Nummerierung und
% Anordnung intern identifiziert.
\MDeclareSiteUXID{BASE_EXAMPLEMODULE}
Dies ist eine generische Einführung in das Modul. Hier empfiehlt es sich, eine Inhaltsangabe
der Unterabschnitte einzustellen. Diese kann man vom Konverter automatisch generieren und
in einer Listenbox ausgeben lassen:

\MModstartBox

% Für bestimmte Worthülsen können Makros verwendet werden, das folgende Makro generiert einen kursweit einheitlichen
% Bezeichner für die Kapitel/Module/Seiten/Lerneinheiten whatever:
In diesem \MSsectionlabelprefix\ wird ein Überblick über die verschiedenen Makros gegeben, die im Makropaket \texttt{mintmod.tex} enthalten sind.
\ \\ \ \\
Mit dem Befehl \texttt{$\backslash$ifttm} kann abgefragt werden, ob eine PDF- oder HTML-Konvertierung erfolgt. In diesem
Fall ist
\ifttm%
ein HTML-Modul
\else%
eine PDF-Datei
\fi
erzeugt worden.
\end{MSectionStart}

\MSubsection{Ein Unterabschnitt}
\MLabel{LABEL_BASE_SUBSECTION_FIRST}
\MDeclareSiteUXID{BASE_SUBSECTION_FIRST}
% Auch Unterabschnitte bekommen eine UXID

% Jegliche Inhalte müssen in MXContent oder MSectionStart-Umgebungen stehen.
% Jede solche Umgebung wird zu einer separaten HTML-Seite.
% Schema: \begin{MXContent}{Titel der Seite}{Kurztitel}{Typ} (Typ ist veraltet und sollte auf STD gesetzt werden)

\begin{MXContent}{Die erste Seite}{Seite 1}{STD}
\MDeclareSiteUXID{BASE_SITE_ONE}
\MLabel{LABEL_BASE_SITE_ONE} % LaTeX-Labels NIEMALS mit \label und \ref sondern mit den gekapselten Versionen deklarieren

Dies ist eine Modulseite. Sie erscheint sowohl in der obigen Button-Auswahlliste als auch in der
Inhaltsliste am linken Rand. Die Liste kann mit dem Menü-Button oben rechts aktiviert werden.

Zur Darstellung mathematischer Formeln wird \texttt{MathJax} eingesetzt:
$$
% Davon sieht man im LaTeX natürlich nichts
\int_0^1 \sin(2\pi x)dx \;=\; \prod_{k=1}^n \left({1-\frac1k}\right)\: .
$$
Für die meisten Stilelemente sind im Rahmen des Onlinebrückenkurses Mathematik des VE\&MINT-Projekts zahlreiche
normative Makros entstanden:
$$
M \;=\;\left\lbrace{x\MCondSetSep x\MBlank\text{erfüllt}\MBlank \text{Bedingung}}\right\rbrace\: .
$$
Die vorhandenen Befehle werden durch MathJax und das Makropaket festgelegt, beispielsweise kann MathJax nichts
mit dem \texttt{$\backslash$texttt}-Befehl in einer Mathematikumgebung anfangen:
$$
\text{Das klappt in der HTML-Übersetzung}\ \texttt{nicht}\: .
$$

Es besteht die Möglichkeit \MEntry{Schlagwörter}{Schlagwort} eintragen zu lassen, die dann über die Suchfunktion gefunden werden können.
% Das kann man auch tun, ohne dass das Wort gedruckt wird
\MIndex{NochEinSchlagwort}
\ \\ \ \\
Diese Seite ist jetzt zuende. Auf die nächste Seite gelangt man mit der Navigation oben oder über diesen \MSRef{LABEL_BASE_SITE_TWO}{Link}.
% Die Links erscheinen auch im PDF
\ \\ \ \\
Man kann aber auch Links \MExtLink{www.mint-kolleg.de}{aus dem Kurs heraus} setzen, natürlich auf eigene Gefahr.
\end{MXContent}

\begin{MXContent}{Die zweite Seite}{Seite 2}{STD}
\MDeclareSiteUXID{BASE_SITE_TWO}
\MLabel{LABEL_BASE_SITE_TWO}
Dies ist die zweite Seite mit einigen Stilelementen. Nochmal zur Erinnerung: Inhalte und Technik sollten getrennt sein,
d.h. der Autor soll sich um die konkrete Ausgestaltung der Boxen nicht scheren, das ist Aufgabe des Modulerstellers:

\begin{MInfo}
Inhaltsteile können farblich hinterlegt werden, z.B. für Definitionen oder Sätze. Diese Boxen dürfen beliebig lang sein.
\end{MInfo}

\begin{MExample}
Dies ist eine Beispielbox. Welche farbliche Ausgestaltung die Boxen haben legt das colorset in den Optionen fest.
\end{MExample}

\begin{MExperiment}
Eine Versuchsbeschreibung in einer weiteren gefärbten Box.
Jede Box wird entsprechend den Moduleinstellungen nummeriert oder nicht.
\end{MExperiment}

\end{MXContent}

\begin{MXContent}{Strukturierung}{Strukturierung}{STD}
    \MDeclareSiteUXID{BASE_SITE_STRUCTURE}
    Die typische Struktur eines Dokuments für \texttt{tex2x} ist
    \ \\
    \ \\
    \begin{itemize}
    \item{\texttt{MSection}\ \ \ \ \textit{Ein Abschnitt}\\
    \texttt{MSectionStart}\ \ \ \ \textit{Einführung/Übersicht}\\
    \begin{itemize}
    \item{\texttt{MSubsection}
    \begin{itemize}
    \item{\texttt{MXContent}\ \ \ \ \textit{Eine konkrete HTML-Seite}}
    \item{\texttt{MXContent}}
    \item{\texttt{...}}
    \end{itemize}
    }
    \end{itemize}
    }
    \end{itemize}
    \ \\
    \begin{MInfo}
    Für die Strukturierung werden folgende Befehle verwenden:\\
    \begin{itemize}
    \item{\verb%\MSection{ABSCHNITTSTITEL}%}
    \item{\texttt{$\backslash$}\verb%begin{MSectionStart} ... \end{MSectionStart}%} % hier muss verhindert werden, dass der Konverter MSectionStart erkennt und übersetzt
    \item{\verb%\MSubsection{UNTERTITEL}%}
    \item{\texttt{$\backslash$}\verb%begin{MXContent}{TITEL}{KURZTITEL}{TYP}% (Abschnittstyp in der Regel \texttt{STD})}
    \item{Alternative Umgebungen statt \texttt{MXContent}: \texttt{MExercises}, \texttt{MTest}}
    \end{itemize}
    \end{MInfo}

    \begin{MExample}
    Der Kurztitel eines \texttt{MXContent} wird dabei für die Navigationsbuttons verwendet:
    \ \\ \ \\
    \begin{center}
    \MUGraphicsSolo{iconsbeispiel.png}{width=0.3\linewidth}{}
    \end{center}
    \end{MExample}

    Diese Strukturelemente sollten stets mit \verb%\MDeclareSiteUXID{KENNUNG}% versehen werden.

    \begin{MInfo}
    Strukturelemente (Abschnitte, Contents, Aufgaben, Boxen etc.) werden auf drei Arten gekennzeichnet:
    \begin{itemize}
    \item{Über Labels, die mit \verb%\MLabel{NAME}% vergeben und und mit \verb%\MRef{NAME}% angesprochen werden. Bis auf die Verlinkung sind die Labels für den Benutzer unsichtbar.}
    \item{Über Nummern, typischerweise in der Form \texttt{X.Y.Z}, die an den Elementen automatisch angebracht wird. Der Konverter vergibt die Nummern automatisch.}
    \item{Über unique extended ids (uxid), diese machen das Element unabhängig von seiner Position im Kurs, seiner Nummer oder seiner Existenz kenntlich und werden vor allem verwendet,
    um eingegebene Antworten zuzuordnen. Eine uxid wird mit \verb%\MDeclareSiteUXID{UXID}% und direkt in den Frageelementen vergeben. Sie muss vom Autor sinnvoll gewählt werden. Unter der
    vergebenen uxid können die eingegebenen Aufgabenlösungen aus der Datenbank abgerufen werden.}
    \end{itemize}
    \end{MInfo}

    \begin{MInfo}
    Schlagwörter werden mit den folgenden Befehlen definiert und in der Suchliste zum gesamten Modul aufgeführt:
    \begin{itemize}
    \item{\verb%\MIndex{SCHLAGWORT}% definiert ein Schlagwort (ohne Ausgabe auf der konkreten Seite),}
    \item{\verb%\MEntry{SCHLAGWORT}{TEXT}% versieht ein Wort aus dem Seitentext (das von dem Schlagwort abweichen kann) mit einem Schlagwort.}
    \end{itemize}
    \end{MInfo}

    \begin{MExample}
    Das Wort \MEntry{Schlagwort}{Schlagwörter} wird unter \texttt{Schlagwörter} in der Liste aufgeführt mit dem Befehl \texttt{$\backslash$}\verb%MEntry{Schlagwort}{Schlagwörter}%.
    Mit dem Befehl \verb%\MIndex{Unwort}% wird \texttt{Unwort} in die Suchliste eingetragen und verlinkt, aber taucht nicht im Ausgabetext auf.
    \end{MExample}

    \end{MXContent}

    \begin{MXContent}{Bilder einbinden}{Bilder einbinden}{STD}
    \MDeclareSiteUXID{BASE_SITE_IMAGES}

    \begin{MInfo}
    Zum Einbinden von Bildern erfolgt mit den Makros
    \begin{itemize}
    \item{\verb%\MGraphics{DATEINAME}{PDF-SKALIERUNG}{TITEL}%}
    \item{\verb%\MUGraphics{DATEINAME}{PDF-SKALIERUNG}{TITEL}{CSS-STYLE}% (dabei können Skalierung/Stile auch leer sein)}
    \item{\verb%\MGraphicsSolo{DATEINAME}{PDF-SKALIERUNG}%}
    \item{\verb%\MUGraphicsSolo{DATEINAME}{PDF-SKALIERUNG}{CSS-STYLE}%}
    \end{itemize}
    \end{MInfo}

    Für die Bilder sind die Formate \texttt{png}, \texttt{jpg}, \texttt{gif}, \texttt{svg} möglich (andere mit Einschränkungen).


    \begin{MExample}
    Eine typische Bildeinbindung mit Nummerierung und Einrückung:
    \MUGraphics{testbild.png}{width=0.3\linewidth}{Dies ist der Bildtitel}{width:400px}
    Sie wurde mit dem Befehl \verb%\MUGraphics{testbild.png}{width=0.3\linewidth}{Dies ist der Bildtitel}{width:400px}%
    erzeugt.
    \end{MExample}

    \begin{MExample}
    Eine einfache Bildeinbindung ohne zusätzliche Elemente, sie wird nicht eingerückt:
    \MUGraphicsSolo{testbild.png}{width=0.2\linewidth}{width:150px}
    Sie wurde mit dem Befehl \verb%\MUGraphicsSolo{testbild.png}{width=0.2\linewidth}{width:150px}%
    erzeugt.
    \end{MExample}


    \end{MXContent}

\begin{MXContent}{Makroliste}{Makroliste}{STD}
\MDeclareSiteUXID{BASE_SITE_MACROS}
Folgende Makros werden im Makropaket des Konverters definiert:
\ \\ \ \\
\MPythonScript{macrodoc.py}
\end{MXContent}

% MExercises ist eine spezialisierte MXContent-Umgebung für Aufgaben, die ein eigenes Icon erhält
\begin{MExercises}
\MDeclareSiteUXID{BASE_SITE_EXERCISES}

Hier ein paar Aufgabenbeispiele:

\begin{MExercise}
Kürzen Sie den Bruch maximal: \MEquationItem{$\frac{22}{11}$}{\MLQuestion{5}{2}{EX_1}}\: .
% Lösungshinweise können mit einer MHint-Umgebung definiert werden,
% diese lassens ich bei Bedarf im ganzen Onlinemodul abschalten
\begin{MHint}{Lösung}
Der größte gemeinsame Teiler von Zähler und Nenner ist $\textrm{ggT}(22,11)=11$, also ist
$$
\frac{22}{11} \;=\; \frac{2}{1} \;=\; 2\: .
$$
\end{MHint}
\end{MExercise}

% Einrückungen werden von LaTeX und dem Konverter ignoriert
\begin{MExercise}
    In welchem reellen Intervall ist die Funktion $f(x)=\sqrt{1-x^2}$ definiert?
    \ \\ \ \\
    % Gewünschte Einrückungen in der Ausgabe müssen mit Zwangsleerzeichen hergestellt werden
    \ \ \ \ \MEquationItem{$D_f$}{%
        \MLIntervalQuestion{10}{[-1,1]}{4}{EX_2}%
    }\: .
    % Hinweise zur Eingabe sollten großzügig eingefügt werden
    \MInputHint{Geschlossene Intervalle können in der Form $[a,b]$, offene Intervalle in der Form $(a,b)$ eingegeben werden.}%
    \ \\
    \begin{MHint}{Im Button muss nicht immer \glqq Lösung\grqq\ stehen}
        Für den Term in der Wurzel muss $1-x^2\geq 0$, also $-1\leq x\leq 1$ gelten.
        Die richtige Antwort ist daher $I=[-1,1]$ als geschlossenes Intervall.
    \end{MHint}
    \begin{MHint}{Alternative Lösung}
        Für $x<0$ ist $|x|=-x$ also negative Intervallseite nicht vergessen!
    \end{MHint}
\end{MExercise}





\end{MExercises}

\MSubsection{Aufgabenelemente}
\MLabel{LABEL_BASE_SUBSECTION_EXERCISEELEMENTS}
\MDeclareSiteUXID{BASE_SUBSECTION_EXERCISEELEMENTS}

\begin{MXContent}{Auswahl- und Zuordnungsfelder}{Auswahl- und Zuordnungsfelder}{STD}
\MDeclareSiteUXID{BASE_SUBSECTION_EXERCISEELEMENTS_INPUTS1}

\begin{MInfo}
Bei Auswahl- und Zuordnungsfeldern wählt der Benutzer zwischen Auswahl- oder Zuordnungsmöglichkeiten aus, von denen eine oder mehrere richtig sein können.
\ \\
\begin{itemize}
\item{\texttt{MLCheckbox\{0 ODER 1\}\{UXID\}}\\
Erstellt eine Tristate-Checkbox mit den Zuständen unberührt/angewählt/abgewählt. Diese werden typischerweise in einer \texttt{MQuestionGroup} (für Multiple-Choice)
oder einer \texttt{MXQuestionGroup} (für Single-Choice) in Form einer Tabelle (beispielsweise \texttt{MDTabular}) gruppiert.}
\item{Umgebung \texttt{MLDropDownQuestion\{UXID\}\{LÖSUNGSTAG\}}\\
Erstellt eine DropDown-Liste, in der Auswahlpunkte über den Befehl \texttt{$\backslash$MDropDownItem\{TEXT\}\{TAG\}} angeordnet werden, jeder mit einem Tag versehen.
Nur bei Auswahl des Items mit dem Lösungstag wird die Auswahl als richtig bewertet.}
\item{Umgebung \texttt{MLDragDropQuestion\{UXID\}\{LÖSUNG\}}\\
Erstellt einen DragDrop-Bereich, indem mit der Maus Kästen gezogen werden können.
Die Kästen werden mit den Befehlen \texttt{MDragDropItem\{TAG\}\{TEXT\}} (für Items die zugeordnet werden können) sowie
\texttt{MDragDropReceptor\{TAG\}\{TEXT\}} (für Behälter die ein Item aufnehmen können) definiert.
Die Lösungszuordnung muss in der Form einer mit dem Semikolon getrennten Liste \texttt{rtag:itag;...;rtag:itag} angegeben werden,
wobei jeweils ein Rezeptor-Tag einem Item-Tag zugeordnet wird. Es ist dabei möglich, dass einige der gelisteten Items nicht zugeordnet werden.}
\end{itemize}

Diese (und auch alle anderen Feldelemente) können in \texttt{MQuestionGroup}-Umgebungen kombiniert werden,
diese werden dann durch einen mit \texttt{$\backslash$MGroupButton\{BUTTONTITEL\}} erzeugten Kontrollbutton ausgewertet.
\end{MInfo}

\begin{MExample}
Eine typische Multiple-Choice-Anordnung in einer Kontrollgruppe mit Button:
\ \\ \ \\
Welche der folgenden Termumformungen ist für $\log(\frac1{100})$ richtig?\\ \ \\
\begin{MQuestionGroup}
\begin{MDTabular}{ll}
\MEquationItem{$\log(\frac1{100})$}{$\frac1{\log(100)}$} & \MLCheckbox{0}{MA_LOGMC_1}\\
\MEquationItem{$\log(\frac1{100})$}{$-\log(100)$} & \MLCheckbox{1}{MA_LOGMC_2}\\
\MEquationItem{$\log(\frac1{100})$}{$100\cdot \log(1)$} & \MLCheckbox{0}{MA_LOGMC_3}\\
\MEquationItem{$\log(\frac1{100})$}{$-100\cdot \log(1)$} & \MLCheckbox{0}{MA_LOGMC_4}\\
\MEquationItem{$\log(\frac1{100})$}{$-2$} & \MLCheckbox{1}{MA_LOGMC_5}\\
\MEquationItem{$\log(\frac1{100})$}{$2\log(\frac1{10})$} & \MLCheckbox{1}{MA_LOGMC_6}
\end{MDTabular}
\end{MQuestionGroup}
\ \\
\MGroupButton{Lösung überprüfen}
\end{MExample}

\begin{MExample}
Das analoge Beispiel als Single-Choice:\ \\ \ \\
Welche der folgenden Termumformungen ist für $\log(\frac1{100})$ richtig?\\ \ \\
\begin{MXQuestionGroup}
\begin{MDTabular}{ll}
\MEquationItem{$\log(\frac1{100})$}{$\frac1{\log(100)}$} & \MLCheckbox{0}{MA_LOGMC_1S}\\
\MEquationItem{$\log(\frac1{100})$}{$-\log(100)$} & \MLCheckbox{1}{MA_LOGMC_2S}\\
\MEquationItem{$\log(\frac1{100})$}{$100\cdot \log(1)$} & \MLCheckbox{0}{MA_LOGMC_3S}\\
\MEquationItem{$\log(\frac1{100})$}{$-100\cdot \log(1)$} & \MLCheckbox{0}{MA_LOGMC_4S}
\end{MDTabular}
\end{MXQuestionGroup}
\ \\
\MGroupButton{Lösung überprüfen}
\end{MExample}

\begin{MExample}
Das DropDown-Feld wird über eine \texttt{MLDropDownQuestion}-Umgebung und darin enthaltene \texttt{MDropDownItem}-Befehle erzeugt:
\ \\
Welche der folgenden Eigenschaften einer Funktion impliziert, dass sie auch stetig ist?
\begin{MLDropDownQuestion}{EX_MLDROPDOWN_1}{lsg}
\MDropDownItem{Monotonie}{fake1}
\MDropDownItem{Differenzierbarkeit}{lsg}
\MDropDownItem{Bijektivität}{fake2}
\end{MLDropDownQuestion}
\end{MExample}

\begin{MExample}
Diese DragDrop-Anordnung wird durch ein \texttt{MLDragDropQuestion}-Umgebung erzeugt:
\ \\ \ \\
Berechnen Sie Schnitt, Vereinigung und Restmenge der beiden Mengen $ A = \{ 1,3,5,7,9\} $ und $B= \{ 3,4,5,6 \}.$ Drei von den sechs möglichen Ergebnissen gehören hier allerdings nicht zur rechten Seite:\ \\
\begin{MLDragDropQuestion}{MA_SETS}{r6:6;r3:3;r4:4}
\MDragDropItem{1}{\{1,3,5,7,9 \}}
\MDragDropItem{2}{\{1,7,8,9\}}
\MDragDropItem{3}{\{1,3,4,5,6,7,9\}}
\MDragDropItem{4}{\{1,7,9\}}
\MDragDropItem{5}{\{4,6\}}
\MDragDropItem{6}{\{3,5\}}
\MDragDropReceptor{r6}{Schnittmenge}
\MDragDropReceptor{r3}{Vereinigungsmenge}
\MDragDropReceptor{r4}{Restmenge}
\end{MLDragDropQuestion}
\end{MExample}

\end{MXContent}

\begin{MXContent}{Einfache Eingabefelder}{Einfache Eingabefelder}{STD}
\MDeclareSiteUXID{BASE_SUBSECTION_EXERCISEELEMENTS_INPUTS0}

\begin{MInfo}
Eingabefelder erlauben dem Benutzer eine textuelle Eingabe einer Lösung. Das Eingabefeld färbt sich (außer in Tests) während der Eingabe passend zur Richtigkeit der Eingabe.
Bei Formeleingabefeldern werden Hilfsboxen für den Benutzer eingeblendet, um die eingegebene Formel darzustellen.
Die Felder akzeptieren typische GTR/CAS/LaTeX-Ausdrücke sowie Wortbeschreibungen wie \texttt{wurzel} oder \texttt{unendlich}.
\begin{itemize}
\item{\texttt{MLQuestion\{ZEICHENZAHL\}\{LÖSUNG\}\{UXID\}}\\
Ein einfaches String-Eingabefeld, dessen Texteingabe nicht interpretiert wird. Mit Komma getrennt können alternative richtige Antworten angegeben werden.
Bei einem Leerstring als Lösung ist jede echte Benutzereingabe richtig.}
\item{\texttt{MLQuestionArea\{SPALTENZAHL\}\{LÖSUNG\}\{ZEILENZAHL\}\{ZEICHENZAHL\}\{UXID\}}\\
Ein mehrzeiliges String-Eingabefeld, dessen Texteingabe nicht interpretiert wird. Mit Komma getrennt können alternative richtige Antworten angegeben werden.
Bei einem Leerstring als Lösung ist jede echte Benutzereingabe richtig.}
\item{\texttt{MLParsedQuestion\{ZEICHENZAHL\}\{LÖSUNG\}\{GENAUIGKEIT\}\{UXID\}}\\
Ein Eingabefeld für Zahlenwerte, diese können als numerische Eingabe oder als Ausdruck eingegeben werden, der zu einer Zahl ausgewertet wird. Die Eingabe ist richtig,
wenn sie bis auf die in \texttt{GENAUIGKEIT} angegebene Anzahl Dezimalstellen hinter dem Dezimaltrenner mit der Lösung übereinstimmt. Falls mehrere mit Komma getrennte
Lösungszahlen angegeben sind erwartet das Eingabefeld vom Benutzer eine Menge als Eingabe.}
\item{\texttt{MLIntervalQuestion\{ZEICHENZAHL\}\{LÖSUNG\}\{GENAUIGKEIT\}\{UXID\}}\\
Ein Eingabefeld für Intervalle, diese sind von der Form $[a,b]$, $(a,b)$ oder halboffen und können als Randpunkte numerische Werte oder Ausdrück erhalten,
die zu Zahlen ausgewertet werden. Die Eingabe ist richtig,
wenn sie bis auf die in \texttt{GENAUIGKEIT} angegebene Anzahl Dezimalstellen hinter dem Dezimaltrenner mit der Lösung übereinstimmt und der Intervalltyp richtig
eingegeben wurde. Der Autor trennt das Intervall stets mit einem Komma, der Benutzer dagegen mit einem Semikolon (oder einem anderen im Kurs
festgelegten Trennzeichen). Für $\infty$ bzw. $-\infty$ können die Wörter \texttt{unendlich}, \texttt{infinity} oder \texttt{infty} (mit oder ohne Backslash davor)
verwendet werden.}
\end{itemize}
\end{MInfo}

Die folgenden Beispiele listen die verfügbaren Eingabefelder auf:

\begin{MExample}
Das \texttt{MLQuestion}-Feld eignet sich für Wörter in Lückentexten:
\ \\ \ \\
Ist $f(x)$ auf $(a,b)$ differenzierbar mit $f(x)>0$ für $a<x<b$, so ist $f$ auf dem Intervall streng monoton \MLQuestion{20}{wachsend,steigend}{EX_MLQuestion1}.
\ \\ \ \\
Das Fragefeld wurde mit dem Kommando \texttt{$\backslash$MLQuestion\{20\}\{wachsend,steigend\}\{EX\_MLQuestion1\}} erzeugt.
\end{MExample}

\begin{MExample}
Das \texttt{MLQuestionArea}-Feld erlaubt Eingaben in mehrzeilige Eingabebereiche. Es arbeitet wie ein \texttt{MLQuestion}-Feld, ist aber hauptsächlich
für nicht wirklich korrigierte Benutzereingaben (z.B. Feedbacks) gedacht. Der Benutzer kann die Größe des Feldes durch Ziehen mit der Maus verändern,
ist aber an die maximale Zeichenzahl gebunden:
\ \\ \ \\
Bitte kommentieren Sie Ihre Antwort auf die vorangehende Frage:\\
\MLQuestionArea{40}{}{3}{100}{EX_MLQuestionArea_1}
\ \\
Das Fragefeld wurde mit dem Kommando \texttt{$\backslash$MLQuestionArea\{40\}\{\}\{3\}\{100\}\{EX\_MLQuestionArea\_1\}} erzeugt.
\end{MExample}

\begin{MExample}
Das \texttt{MLParsedQuestion}-Feld ist das Standardeingabefeld für numerische Eingaben, Lösung und Eingaben können als numerischer Wert oder als auswertbarer Ausdruck eingegeben werden:
\begin{itemize}
\item{Was ist die Wurzel aus $36$? Antwort: \MEquationItem{$\sqrt{36}$}{\MLParsedQuestion{3}{wurzel(36)}{5}{EX_MLParsedQuestion1}}.\\
Das Fragefeld wurde mit dem Kommando \texttt{$\backslash$MLParsedQuestion\{3\}\{wurzel(36)\}\{5\}\{EX\_MLParsedQuestion1\}} erzeugt.}
\item{Ein Körper der Masse $m=2$ Kilogramm besitzt auf der Erde ($g=9.81$) die Fallbeschleunigung \MEquationItem{$a$}{\MLParsedQuestion{20}{2*9.81}{5}{EX_MLParsedQuestion2}}.\\
Das Fragefeld wurde mit dem Kommando \texttt{$\backslash$MLParsedQuestion\{20\}\{2*9.81\}\{5\}\{EX\_MLParsedQuestion2\}} erzeugt, es akzeptiert sowohl die numerische
Lösung wie auch Formeleingaben oder äquivalente Ausdrücke bis auf Rundung zur 5. Nachkommastelle, d.h.
\ \\
\begin{itemize}
\item{\texttt{19.62} und \texttt{19,62} sind richtige Eingaben (der Autor gibt seine Lösung dagegen stets mit Dezimalpunkt ein),}
\item{\texttt{2*9.81} ist eine richtige Eingabe,}
\item{\texttt{19,62001} ist eine falsche Eingabe,}
\item{\texttt{19,620001} ist eine richtige Eingabe,}
\item{\texttt{19,620005} ist eine falsche Eingabe (unterscheidet sich nach Rundung zur 5. Stelle von der Lösung.}
\end{itemize}
}
\end{itemize}
Tipp: Auch die Lösung wird ausgewertet, zur Fehlervermeidung empfiehlt es sich den Lösungswert nicht separat auszurechnen und direkt einzutragen. Über die Einstellung
der Feldlänge kann die Verwendung der Termlösung für den Benutzer unterbunden werden, alternativ kann \texttt{MLSimplifyQuestion} eingesetzt werden
um in beliebig langen Eingabefeldern die Verwendung von Termen/Funktionen in der Benutzereingabe zu unterdrücken.
\end{MExample}

\begin{MExample}
Das Zahleneingabefeld kann genutzt werden, um symbolische Eingaben vom Benutzer auszuwerten:
\ \\ \ \\
In einem rechtwinkligen Dreieck habe die Ankathete an einem Winkel $\alpha=\frac13\pi$ die Länge $5$, welche Länge besitzt die Hypothenuse?\ \\ \ \\
Die Länge der Hypothenuse beträgt \MEquationItem{$c$}{\MLParsedQuestion{20}{5/(cos(pi/3))}{25}{EX_MLParsedQuestion3}}.\ \\ \ \\
Das Fragefeld wurde mit dem Kommando \texttt{$\backslash$MLParsedQuestion\{20\}\{5/(cos(pi/3))\}\{25\}\{EX\_MLParsedQuestion3\}} erzeugt.
Für eine Genauigkeit von beispielsweise 25 statt 4 Stellen würde $10$ nicht als Eingabe akzeptiert werden, da der Kosinus mithilfe eines Taylorpolynoms achten
Grades berechnet wird und in der Musterlösung Rundungsfehler auftreten.
\ \\
\begin{itemize}
\item{\texttt{5/cos(1/3*pi)} oder auch \texttt{5/cos($\backslash$pi/3)} sind richtige Eingaben.}
\item{Vorsicht ist bei hohen Genauigkeiten in der Lösung geboten: \texttt{5/(sin(pi/2+1/3*pi))} ist hier keine richtige Eingabe, weil durch Rundungsfehler im Browser (JavaScript)
nicht $\cos(x)=\sin(x+\frac{\pi}{2})$ gegeben ist.}
\item{Genauigkeit und maximale Zeichenzahl verhindern, dass der Benutzer den Wert per Taschenrechner numerisch bestimmt und eingibt, auch wenn es hier eine ganze Zahl ist.}
\item{Das Fragefeld kann natürlich nicht erkennen, dass $c=-20$ hier eine numerisch richtige aber mathematisch unsinnige Lösung auf eine inkorrekte Fragestellung ist.}
\end{itemize}
\end{MExample}

\begin{MExample}
Das Zahleneingabefeld kann zur Erkennung von Mengen eingesetzt werden. Dabei wird jedes Mengenelement numerisch ausgewertet:
\ \\ \ \\
Die Nullstellenmenge von $f(x)=x^3-3x^2$ ist \MLParsedQuestion{20}{0,3}{5}{EX_MLParsedQuestion4}\: .
\ \\ \ \\
Das Fragefeld wurde mit dem Kommando \texttt{$\backslash$MLParsedQuestion\{20\}\{0,3\}\{5\}\{EX\_MLParsedQuestion4\}} erzeugt.
\ \\
\begin{itemize}
\item{Die Eingaben \texttt{\{0;3\}} sowie \texttt{\{3;0\}} aber auch \texttt{\{3;0;3;0;0\}} sind richtig.}
\item{Die Eingabe \texttt{\{0,3\}} ist hier nicht richtig (Trennzeichen Semikolon für Nutzer kann im Kurs eingestellt werden, es ist aber stets das Komma für den Autor).}
\item{Die Eingabe \texttt{\{1-1;sqrt(9)\}} ist richtig.}
\item{Die leere Menge \texttt{\{\}} sowie die Eingaben \texttt{0,3} bzw. \texttt{[0;3]} oder andere Aufzählungstechniken sind dagegen nicht richtig.}
\end{itemize}
\end{MExample}

\begin{MExample}
Das Intervallfragefeld kann für numerische wie für algebraisch angegebene Randpunkte eingesetzt werden:
\ \\ \ \\
Auf welchem Intervall ist die Ungleichung $1<x^2\leq 2$ erfüllt?\ \ \MLIntervalQuestion{20}{(1,sqrt(2)]}{5}{EX_MIntervalQuestion1}.
\ \\ \ \\
Dieses Fragefeld wurde mit dem Befehl \texttt{$\backslash$MLIntervalQuestion\{20\}\{(1,sqrt(2)]\}\{5\}\{EX\_MIntervalQuestion1\}} erzeugt.
\end{MExample}

\end{MXContent}


\begin{MXContent}{Eingabefelder für Terme}{Eingabefelder für Terme}{STD}
\MDeclareSiteUXID{BASE_SUBSECTION_EXERCISEELEMENTS_INPUTS2}

\begin{MInfo}
Eingabefelder für Terme spezifizieren neben dem Term eine Liste von Auswertungsvariablen. Die Lösungskontrolle erfolgt durch Auswertung des Terms an (ggf. mehrdimensionalen) Stützstellen
zur vorgegebenen Genauigkeit (nach Ende der Auswertung). Es werden sehr viele verschiedene Ausdrücke in verschiedenen Notationen akzeptiert, diese sind in der Datei \texttt{src/files/mparser.js}
codiert. Auch erweiterte Konstruktionen wie Summen, Produkte und Integrale stehen zur Verfügung.
\ \\ \ \\
Termeingaben werden über diese beiden Makros eingerichtet:
\begin{itemize}
\item{\texttt{MLFunctionQuestion\{ZEICHENZAHL\}\{LÖSUNG\}\{STÜTZSTELLENZAHL\}\{VARIABLENLISTE\}\{GENAUIGKEIT\}\{UXID\}}\\
Die Variablen sind nicht case-sensitive und werden durch Komma getrennt gelistet. Die Variablenliste kann auch leer sein, dann erwartet das Feld eine Eingabe die einen konkreten Zahlenwert
ergibt.}
\item{\texttt{MLSimplifyQuestion\{ZEICHENZAHL\}\{LÖSUNG\}\{STÜTZSTELLENZAHL\}\{VARIABLENLISTE\}\{GENAUIGKEIT\}\{TYP\}\{UXID\}}\\
Der Vereinfachungstyp legt fest, welche Form der Ausdruck zusätzlich zur numerischen Richtigkeit aufweisen muss, um als richtige Lösung akzeptiert zu werden.}
\end{itemize}
Bei diesen Fragefeldtypen erscheint während der Benutzereingabe ein Popup-Fenster, das den eingegebenen Ausdruck mathematisch darstellt und
Tipps zur Eingabe gibt (beispielsweise dass \texttt{6*x} statt \texttt{6x} einzugeben ist).
\end{MInfo}

Der Typ einer Vereinfachungsaufgabe mit \texttt{MLSimplifyQuestion} ist dabei die Summe aus den diesen Grundtypen
\ \\
\begin{itemize}
\item{\texttt{0}: Keine Vereinfachung gefordert, bis auf Optionen gleiche Wirkung wie bei \texttt{MLFunctionQuestion}}
\item{\texttt{1}: Keine Klammern (runde oder eckige) dürfen im vereinfachten Ausdruck auftreten}
\item{\texttt{2}: Faktordarstellung (Term hat Produkte als letzte Operation, Summen als vorgeschaltete Operation) \textit{(noch nicht verfügbar)}}
\item{\texttt{3}: Summendarstellung (Term hat Summen als letzte Operation, Produkte als vorgeschaltete Operation) \textit{(noch nicht verfügbar)}}
\end{itemize}
sowie beliebig vielen der folgenden Optionen:
\begin{itemize}
\item{\texttt{16}: Maximal ein slash (Bruchstrich) im Ausdruck gestattet}
\item{\texttt{32}: Stammfunktion ist gefragt, Eingabe muss als Term in einer Variablen vorliegen und wird nur modulo Konstanten bewertet (die vom Fragefeld automatisch zu f(0)=0 normiert wird)}
\item{\texttt{64}: Keine Wurzelfunktion erlaubt (kann aber noch als x\^{}(1/2) geschrieben werden)}
\item{\texttt{128}: Keine Betragsfunktion erlaubt (kann aber noch als Fallunterscheidung geschrieben werden)}
\item{\texttt{256}: Keine Brüche und keine Potenzen erlaubt (also kein \texttt{/} und kein \texttt{\^{}})}
\item{\texttt{512}: Besondere Stützstellen für alle Variablen ($>1$ und schwach rational, sonst symmetrisch um Nullpunkt und ganze Zahlen)}
\item{\texttt{1024}: Nur String aus Ziffern $0\ldots 9$ in Lösung erlaubt}
\item{\texttt{2048}: Höchstens ein \texttt{\^{}} und weder \texttt{/} noch \texttt{*} erlaubt}
\end{itemize}


\begin{MExample}
Das Funktionseingabefeld kann genutzt werden, um einfache Funktionsausdrücke abzufragen:
\ \\ \ \\
Angenommen $f:\:\R\rightarrow\R$ ist stetig differenzierbar mit $f'(x)=12x+1$ und $f(1)=1$, dann ist\ \\ \ \\
\MEquationItem{$f(x)$}{\MLFunctionQuestion{20}{6*x^2+x-6}{5}{x}{5}{EX_MLFunctionQuestion1}}.
\ \\ \ \\
Das Fragefeld wurde mit dem Kommando \texttt{$\backslash$MLFunctionQuestion\{20\}\{6*x\^{}2+x-6\}\{5\}\{x\}\{5\}\{EX\_MLFunctionQuestion1\}} erzeugt. Bei der Eingabe erhält der Benutzer einen
Hinweis im Popup-Fenster, falls er \texttt{6x} statt \texttt{6*x} eingibt. Das Feld erkennt das Quadrat in der Form \texttt{x*x}, \texttt{x\^{}2} oder \texttt{x}$^2$ (das 2-Potenzzeichen, das
man auf Windows-Tastaturen mit AltGr-2 erhält oder durch Copy\&Paste von der Kursseite bekommen kann).
\end{MExample}

\begin{MExample}
Die Eingabe bei Vereinfachungsaufgaben macht nur mit Restriktionen Sinn, da ansonsten auch der Term aus der Aufgabenstellung eine richtige Lösung wäre:
\ \\ \ \\
Multiplizieren Sie vollständig aus: \MEquationItem{$(a+b)(1+x)$}{\MLSimplifyQuestion{40}{(a+b)*(1+x)}{5}{a,b,x}{5}{257}{EX_MLSimplifyQuestion1}}.
\ \\ \ \\
Das Fragefeld wurde mit dem Kommando \texttt{$\backslash$MLSimplifyQuestion\{40\}\{(a+b)*(1+x)\}\{5\}\{a,b,x\}\{5\}\{EX\_MLSimplifyQuestion1\}} erzeugt.
Bei der Eingabe erhält der Benutzer einen Hinweis wenn er die geforderten Vereinfachungsregeln ($1$: Keine Klammern, $256$: Keine Brüche/Potenzen) verletzt.
\begin{itemize}
\item{Eingabe \texttt{a*(1+x)+b*(1+x)} wird als falsch gewertet, es gibt aber einen Hinweis, dass der Term richtig ist und nur die Vereinfachung fehlt.}
\item{Eingabe \texttt{a+a*x+b+b*x} oder jede andere Anordnung der Summanden ist richtig.}
\item{Eingabe von \texttt{a+a*x+b+b*x+log(x)-log(x)} wird wegen der Klammern nicht akzeptiert.}
\item{Eingabe von \texttt{a+a*x+b+b*x+1-1} wird gemäß den gewünschten Vereinfachungsregeln (kein minimaler Term verlangt) akzeptiert.}
\end{itemize}
\end{MExample}

\begin{MExample}
Die Funktionseingabefelder erlauben die Eingabe von Sequenzkonstrukten (Summe, Produkt, Integral) in verschiedenen Notationen (für Autor und Bearbeiter):
\ \\ \ \\
Geben Sie einen symbolischen Ausdruck ohne die Punkte-Schreibweise an:
\ \\ \ \\
\MEquationItem{$a\cdot a^2\cdot a^3\cdot\cdots\cdot a^{100}$}{\MLSimplifyQuestion{30}{prod_(k=1)^(100)a^k}{5}{a}{5}{0}{EX_MLSimplifyQuestion2}}.
\ \\ \ \ \\
Das Fragefeld wurde mit dem Kommando \texttt{$\backslash$MLSimplifyQuestion\{30\}\{prod\_(k=1)\^{}(100)a\^{}k\}\{5\}\{a\}\{5\}\{0\}\{EX\_MLSimplifyQuestion2\}} erzeugt.
 \ \\
\begin{itemize}
\item{Die Eingaben \texttt{prod\_(k=1)\^{}(100) a\^{}k} oder ähnliche Schreibweisen mit \texttt{produkt} oder \texttt{$\backslash$prod} werden akzeptiert, der Name der Laufvariablen ist frei wählbar (wobei sie nicht in der Variablenliste aufgeführt werden darf).}
\item{Die Teilausdrücke für den Laufindex müssen mit runden Klammern geklammert werden, wenn es mehr als ein Symbol ist.}
\item{Das Produkt wird tatsächlich ausgewertet, d.h. hier fallen $100$ Faktoren an die berechnet werden müssen.}
\item{Die Eingabe $a^{5050}$ wird auch als richtig erkannt.}
\end{itemize}
\end{MExample}

\end{MXContent}

\begin{MXContent}{Spezialfelder}{Spezialfelder}{STD}
\MDeclareSiteUXID{BASE_SUBSECTION_EXERCISEELEMENTS_INPUTS3}

\begin{MInfo}
Das Kommando \texttt{$\backslash$MLSpecialQuestion\{ZEICHENZAHL\}\{LÖSUNG\}\{STÜTZSTELLENZAHL\}\{VARIABLE\}\{GENAUIGKEIT\}\{TYP\}\{UXID\}}
erzeugt ein spezielles Eingabefeld, dessen Korrektur durch den String \texttt{TYP} bestimmt wird. Es deckt Spezialfälle ab, die mit den
Standardeingabefeldern nicht oder nur sehr umständlich realisiert werden können. Es gibt diese Typen:
\begin{itemize}
\item{Typ \texttt{onlyempty}: Eingabe ist richtig, wenn der Benutzer \textit{nichts} (den leeren String) eingetragen hat.}
\item{Typ \texttt{evennat}: Eingabe ist richtig, wenn der Benutzer eine beliebige gerade natürliche Zahl eingegeben hat.}
\item{Typ \texttt{oddnat}: Eingabe ist richtig, wenn der Benutzer eine beliebige ungerade natürliche Zahl eingegeben hat.}
\item{Typ \texttt{intervalelement}: Eingabe ist richtig, wenn der Benutzer ein Element aus dem in \texttt{LÖSUNG} angegebenen Intervall eingegeben hat.}
\item{Typ \texttt{exactfraction}: Eingabe ist richtig, wenn der Benutzer einen einfachen maximal gekürzten Bruch mit nichtnegativem Nenner ohne Klammern eingegeben hat, dessen Wert mit der \texttt{LÖSUNG} übereinstimmt.}
\item{Typ \texttt{inputstring2}: Eingabe ist richtig, wenn der Benutzer einen String der Länge 2 (egal welchen) eingegeben hat (macht nur in Codefeldern für Feedbacks Sinn).}
\item{Typ \texttt{inputnumber2}: Eingabe ist richtig, wenn der Benutzer einen Zahlenstring der Länge 2 (egal welchen) eingegeben hat (macht nur in Codefeldern für Feedbacks Sinn).}
\end{itemize}
\end{MInfo}

\begin{MExample}
Leere Eingabelösungen eignen sich für Kombinationen von Eingabefeldern zu einer Aussage:
\ \\ \ \\
Geben Sie jeweils einen beliebigen Wert für die gefragte Variable an, so dass die Gleichung
$$
(y-x)^n\;=\; (x-y)^n
$$
immer (d.h. für alle möglichen Werte der anderen Variablen) erfüllt ist oder lassen Sie die Eingabe leer, falls es nicht möglich ist:
\begin{MQuestionGroup}
\begin{itemize}
% bug: empty-Felder werden trotz group beim Laden ausgewertet
\item{Variable $x\in\R$: Beispielsweise \MEquationItem{$x$}{\MLSpecialQuestion{5}{}{2}{}{2}{onlyempty}{EX_MLSimplifyQuestion0a}}}
\item{Variable $y\in\R$: Beispielsweise \MEquationItem{$y$}{\MLSpecialQuestion{5}{}{2}{}{2}{onlyempty}{EX_MLSimplifyQuestion0b}}}
\item{Variable $n\in\N$: Beispielsweise \MEquationItem{$n$}{\MLSpecialQuestion{5}{}{2}{}{2}{evennat}{EX_MLSimplifyQuestion0c}}}
\end{itemize}
\end{MQuestionGroup}
\MGroupButton{Lösung kontrollieren}
\end{MExample}

\begin{MExample}
Dieses Eingabefeld kann man nur mit \texttt{MLSpecialQuestion} erstellen:
\ \\ \ \\
Finden Sie irgend eine Lösung der Ungleichung $|x-5|<1$. Antwort: Beispielsweise \MEquationItem{$x$}{\MLSpecialQuestion{10}{(4,6)}{1}{}{5}{intervalelement}{EX_MLSpecialQuestion1}}.
\ \\ \ \\
Das Feld wurde mit dem Befehl \texttt{$\backslash$MLSpecialQuestion\{10\}\{(4,6)\}\{1\}\{\}\{5\}\{intervalelement\}\{EX\_MLSpecialQuestion1\}} erzeugt.
\end{MExample}

\begin{MExample}
Exakte Kürzung kann nur mit \texttt{MLSpecialQuestion} geprüft werden:
\ \\ \ \\
Kürzen Sie den Bruch $\frac{42}{-12}$ maximal. Antwort: \MEquationItem{$\frac{42}{-12}$}{\MLSpecialQuestion{10}{42/(-12)}{1}{}{5}{exactfraction}{EX_MLSpecialQuestion2}}.
\ \\ \ \\
Das Feld wurde mit dem Befehl \texttt{$\backslash$MLSpecialQuestion\{10\}\{42/(-12)\}\{1\}\{\}\{5\}\{exactfraction\}\{EX\_MLSpecialQuestion2\}} erzeugt.
\ \\ \ \\
Hier ist \texttt{-7/2} die einzige richtige Antwort. Sowohl \texttt{7/(-2)} wie auch \texttt{(-7)/2} werden abgelehnt.
\end{MExample}

\begin{MExample}
Die Stringlängenfelder sind für Feedbackcodes konzipiert:
\ \\ \ \\
Geben Sie für die Verfolgung Ihrer Teilnahmedaten Ihren Anonymisierungscode ein, gebildet aus den ersten beiden Buchstaben Ihres Geburtsorts
und den letzten beiden Ziffern jetzigen Postleitzahl:
\ \\ \ \\
\MEquationItem{Code}{\MLSpecialQuestion{2}{}{1}{}{5}{inputstring2}{EX_MLSpecialQuestion3a}\MLSpecialQuestion{2}{}{1}{}{5}{inputnumber2}{EX_MLSpecialQuestion3b}}.
% bug: inputnumber2 akzeptiert ZAHL-BUCHSTABE
\end{MExample}

\end{MXContent}

\begin{MXContent}{Roulette Aufgaben}{Roulette Aufgaben}{STD}
\MDeclareSiteUXID{BASE_SITE_EXERCISEELEMENTS_ROULETTE}

Mit den folgenden Aufgabentypen werden Aufgaben aus Python Skripten erzeugt. So ist es z.B.
möglich parametrisierte Aufgabenreihen zu erstellen. Im folgenden werden die verschiedenen
Rouletteaufgabentypen vorgestellt.

\MTitle{Autoroulette}
Autoroulleteaufgaben beziehen ihre Inhalte aus einem python script, das TeX zurückgibt, als beispiel, siehe parabelaufgabe.py in
\begin{verbatim}
tub_base/de/parabelaufgabe.py.
\end{verbatim}

\MTitle{Roulette Typ 0:}
Typ 0 Autorouletteaufgaben bieten einen Button, der eine zufällige neue Aufgabe erzeugt. Dabei wird darauf geachtet, dass bereits angezeigte Aufgaben erst wieder erscheinen, wenn
Nutzer*innen alle anderen Aufgaben angezeigt wurden.
Die folgende Aufgabe wurde mit dem Befehl
\begin{verbatim}
\MAutoRouletteExercises {parabelaufgabe.py}{5}{Roulette0}{0}
\end{verbatim}
erzeugt.
% Wende die generate-Funktion aus agex_miniaufgabe.py 3-mal an um Aufgaben zu generieren (vom Typ 1: In Reihenfolge und mit Durchklickbuttons)
\MAutoRouletteExercises{parabelaufgabe.py}{5}{Roulette0}{0}

\MTitle{Roulette Typ 1:}
Typ 1 Autorouletteaufgaben ermöglichen es Benutzer*innen durch alle Aufgaben zu navigieren.
Die folgende Aufgabe wurde mit dem Befehl
\begin{verbatim}
\MAutoRouletteExercises {parabelaufgabe.py}{5}{Roulette1}{1}
\end{verbatim}
erzeugt.
\MAutoRouletteExercises{parabelaufgabe.py}{5}{Roulette1}{1}

\MTitle{Roulette Typ 2:}
Typ 2 Autorouletteaufgaben bieten immer nur eine Aufgabe der Aufgabenserie an und können so z.B. für Tests benutzt werden.
Die folgende Aufgabe wurde mit dem Befehl
\begin{verbatim}
\MAutoRouletteExercises {parabelaufgabe.py}{5}{Roulette2}{2}
\end{verbatim}
erzeugt.
\MAutoRouletteExercises{parabelaufgabe.py}{5}{Roulette2}{2}

\MTitle{Directroulette}
Directrouletteaufgaben beziehen ihre Inhalte aus TeX-Dateien. Als Beispiel, siehe
\MDirectRouletteExercises{fraction.rtex}{VBKM01_FRACTIONTRAINING}{0}

\end{MXContent}

\begin{MTest}{Abschlusstest Kapitel 1}
\MDeclareSiteUXID{VBKM01_Abschlusstest}
\begin{MExercise}
\MSetPoints{1}
Kreuzen Sie an, ob diese mathematischen Ausdr"ucke jeweils Gleichungen, Ungleichungen, Terme oder Zahlen darstellen (Mehrfachnennung ist m"oglich):
\ \\
\begin{tabular}{|l|c|c|c|c|}
  \hline
  Mathematischer Ausdruck  & Gleichung & Ungleichung & Term & Zahl \\ \hline
  $1+\Mdfrac12-3(3-\Mdfrac12)$ & \MLCheckbox{0}{TX11} & \MLCheckbox{0}{TX12} &\MLCheckbox{1}{TX13} &\MLCheckbox{1}{TX14} \\ \hline
  $5^x-x^5$                & \MLCheckbox{0}{TX21} & \MLCheckbox{0}{TX22} &\MLCheckbox{1}{TX23} &\MLCheckbox{0}{TX24} \\ \hline
  $x^2<\sqrt{x}$           & \MLCheckbox{0}{TX41} & \MLCheckbox{1}{TX42} &\MLCheckbox{0}{TX43} &\MLCheckbox{0}{TX44} \\ \hline
  $x y z-1$                  & \MLCheckbox{0}{TX31} & \MLCheckbox{0}{TX32} &\MLCheckbox{1}{TX33} &\MLCheckbox{0}{TX34} \\ \hline
  $b^2=4a c$               & \MLCheckbox{1}{TX51} & \MLCheckbox{0}{TX52} &\MLCheckbox{0}{TX53} &\MLCheckbox{0}{TX54} \\ \hline
\end{tabular}
\end{MExercise}

\begin{MExercise}
Vereinfachen Sie den Doppelbruch $\Mdfrac{3+\Mtfrac32}{\Mtfrac1{12}+\Mtfrac14}$ zu einem gek"urzten Einfachbruch: \MLParsedQuestion{5}{27/2}{5}{ER6}
\MInputHint{Tippen Sie beispielsweise $\Mdfrac{11}{12}$ als \texttt{11/12} ein.}
\end{MExercise}

\begin{MExercise}
Multiplizieren Sie diesen Term vollst"andig aus und fassen Sie zusammen:

$(x-1)\cdot(x+1)\cdot(x-2)$ = \MLSimplifyQuestion{40}{(x-1)*(x+1)*(x-2)}{10}{x}{5}{1}{SMPPOLY}.

\MInputHint{Beispielsweise tippen Sie $(x+1)(x+2)$ = \texttt{x^2+3*x+2} oder auch \texttt{x*x+3*x+2}.}
\end{MExercise}

\begin{MExercise}
Wenden Sie jeweils eine binomische Formel an, um den Term umzuformen:
\begin{MExerciseItems}
\item{$(x-3)(x+3)$ = \MLSimplifyQuestion{14}{(x-3)*(x+3)}{10}{x}{5}{1}{VBK1}.}
\item{$(x-1)^2$ = \MLSimplifyQuestion{14}{(x-1)*(x-1)}{10}{x}{5}{1}{VBK2}.}
\item{$(2x+4)^2$ = \MLSimplifyQuestion{14}{(2*x+4)*(2*x+4)}{10}{x}{5}{1}{VBK3}.}
\end{MExerciseItems}
\MInputHint{Beispielsweise tippen Sie $(x+1)^2$ = \texttt{x^2+2*x+1} oder auch \texttt{x*x+2*x+1}.}
\end{MExercise}

\begin{MExercise}
Schreiben Sie diesen Potenz- und Wurzelausdruck als einfache Potenz mit einem rationalen Exponenten:
\ \\ \ \\
\MEquationItem{$\Mdfrac{x^3}{\left({\sqrt{x}}\right)^3}$}{\MLSimplifyQuestion{10}{x^(3/2)}{10}{x}{5}{576}{VBK4}}.\\
\MInputHint{Beispielsweise tippen Sie $\sqrt{x}\cdot x^2$ = \texttt{x^(5/2)} oder auch \texttt{x^(2.5)},\\vergessen Sie die Klammern um den Bruch nicht.}
\end{MExercise}

\MTitle{Roulette}
\MAutoRouletteExercises{parabelaufgabe.py}{5}{Roulette2}{2}


\end{MTest}


\MSubsection{Individuelle Anpassungen}
\MDeclareSiteUXID{BASE_SUBSECTION_MODIFICATION}

\begin{MXContent}{HTML5 und JavaScript im Quellcode}{HTML5 und JavaScript im Quellcode}{STD}
\MLabel{LABEL_BASE_SUBSECTION_MODIFICATION_HTMLJS}
\MDeclareSiteUXID{BASE_SUBSECTION_MODIFICATION_HTMLJS}

\begin{MInfo}
An jeder Stelle im Quelldokument kann die LaTeX-Konvertierung umgangen und direkt HTML5-Code eingefügt werden
mithilfe des Kommandos \texttt{$\backslash$special\{html:\textit{HTML-Code}\}} (auch inline) oder mit der \texttt{html}-Umgebung.
\ \\ \ \\
Dabei sollte stets mit \texttt{$\backslash$ifttm} die HTML-Übersetzung abgefragt und im anderen Fall (PDF) ein Ersatztext angegeben werden.
\end{MInfo}

\begin{MExample}
\ifttm
Dieses \special{html:<div style="background-color:rgb(255,0,0);">gefärbte&nbsp;div</div>} erscheint im HTML mit Hilfe
des Befehl \texttt{$\backslash$special\{html:<div style="background-color:rgb(255,0,0);{"}>gefärbte\&nbsp;div</div>\}}.
Wie man sieht sind im Codebereich für HTML die üblichen LaTeX-Umsetzungen auch für Sonderzeichen deaktiviert.
\else
In der HTML-Ausgabe erscheint hier ein inline-div, aber im PDF kann man nur einen Ersatztext wie diesen schreiben.
\fi
\end{MExample}

\begin{MExample}
Mit der \texttt{html}-Umgebung lassen sich größere HTML5/JS/jQuery-Codestücke einbauen, hier beispielsweise ein eigenes
DOM-Element mit etwas JS-Code:
\ \\
\ifttm
\begin{html}
<button id="myangrybutton" type="button" onClick="myButtonClick();" >Click mich Click mich Click mich JETZT</button> (kann man tatsächlich)
<script>
function myButtonClick() {
    alert("Es wurde geclickt!");
}
</script>
\end{html}
\else
Wieder gibt es im PDF nichts zu sehen.
\fi
\ \\
Dabei steht im LaTeX folgender Code:\\
\begin{verbatim}
\\begin{html}
<button id="myangrybutton" type="button" onClick="myButtonClick();" >Click mich Click mich Click mich JETZT</button> (kann man tatsächlich)
<script>
function myButtonClick() {
    alert("Es wurde geclickt!");
}
</script>
\end{html}
\end{verbatim}
\end{MExample}

\begin{MExample}
Im Optionen-Objekt (für dieses Modul ist es \texttt{Option\_BASE\_testing.py}) können zusätzliche JS-Dateien deklariert werden.
Diese werden dann in \textit{jeder} Seite des HTML-Baums eingebunden. Falls nur eine Einbindung auf der aktuellen Seite gewünscht ist,
kann man dies mit dem Befehl
\texttt{$\backslash$special\{html:<script srcx="{}$:$directmaterial$:$dateiname.js"{} type="{}text/javascript"{}></script>\}}
vornehmen. Dabei ist zu beachten, dass die JavaScript-Datei mit dem Befehl \texttt{$\backslash$MRegisterFile\{DATEINAME\}} registriert
und der Dateiname mit dem Marker \texttt{$:$directmaterial$:$} versehen werden
muss, damit sie vom Konverter im HTML-Baum an der richtigen Stelle eingesetzt wird.
Auch wurde das \texttt{src}-Attribut im Einbindungsbefehl im LaTeX-Quellcode-Beispiel verstellt zu \texttt{srcx}, damit das Auftreten des Dateinamens
nicht vom Konverter gefunden und ersetzt wird.
Hier ein Beispiel für die Benutzung einer lokalen JS-Datei:
\ifttm
\begin{html}
<script src=":directmaterial:eintestscript.js" type="text/javascript">
</script>
<div id="mylittlediv">
</div>
<script>
doTheThing();
</script>
\end{html}
% Die Datei liegt im Quellbereich im top-Ordner, wird aber in der HMTL-Ausgabe in den Ordner der generierten HTML-Datei gelegt
\MRegisterFile{eintestscript.js}
\else
Schon wieder kann man es im PDF nicht ausprobieren.
\fi
\ \\
Dabei wurden im LaTeX-Code die Zeilen
\begin{verbatim}
\\begin{html}
<script src=":directmaterial:eintestscript.js" type="text/javascript">
</script>
<div id="mylittlediv">
</div>
<script>
doTheThing();
</script>
\end{html}
\end{verbatim}
ausgeführt und in der JavaScript-Datei steht
\begin{verbatim}
function doTheThing() {
    $('<p>Dieser Text wurde über eine jQuery-Funktion eingefüllt!</p>').appendTo('#mylittlediv');
}
\end{verbatim}
\end{MExample}

\end{MXContent}

\begin{MXContent}{Pythoncode einbinden}{Pythoncode einbinden}{STD}
\MDeclareSiteUXID{BASE_SITE_PYTHON}

\begin{MExample}
Mit dem Befehl \texttt{$\backslash$MPythonScript\{DATEINAME\}} kann ein Pythonskript ausgeführt werden, seine Ausgabe wird als LaTeX-Code interpretiert
und an Stelle des Kommandos eingesetzt. Hier ein Beispiel: \MPythonScript{blabla.py}
\ \\ \ \\
Diese Ausgabe wurde mit folgendem Python-Script erzeugt:
\begin{verbatim}
import numpy as np
import matplotlib.pyplot as plt

# Erstellung eines Plots mit matplotlib, dieser wird dann gespeichert und als Grafik im LaTeX-Code eingebunden
ax = plt.subplot(111)
t = np.arange(-6.0, 6.0, 0.02)
s = (t - 2) * (t - 1) * t * (t + 1) * (t + 2)
line, = plt.plot(t, s, lw = 2)
fname = "pybild.png"
plt.savefig(fname, transparent = True, dpi = 600)
plt.clf()
plt.cla()
plt.close()

print("Hier ein Python-erzeugtes richtig hochauflösendes Bild:\\\\\\MUGraphicsSolo{" + fname + "}{width=0.8\\linewidth}{width:400px}\\ \\\\")
print("Und etwas schleifengeneriertes BlaBla:\\")
for k in range(100):
    print("Bla" + str(k) + " ")
\end{verbatim}
\ \\ \ \\
Der Präprozessor erzeugt daraus folgenden LaTeX-Code, der von pdflatex sowie ttm umgesetzt wird:
\ \\
\begin{verbatim}
\\\MUGraphicsSolo{pybild.png}{width=0.8\linewidth}{width:400px}\ \\
Und etwas schleifengeneriertes BlaBla:\
Bla0
Bla1
Bla2
...usw
\end{verbatim}
\end{MExample}

\end{MXContent}

\begin{MXContent}{Einbinden von Videos}{Einbinden von Videos}{STD}
\MDeclareSiteUXID{BASE_SITE_VIDEOS}

Mit dem Befehl \texttt{$\backslash$MYoutubeVideo\{TITEL\}\{BREITE\}\{HÖHE\}\{URL\}} kann ein embedded-Youtubevideo eingebunden werden, dazu wird ein von YouTube erzeugter
Player auf der Seite eingeblendet, der bei Bedarf vergrößert werden kann.
\ \\
\MYoutubeVideo{Newtons Laws (2)}{400}{300}{https://www.youtube.com/embed/WzvhuQ5RWJE?rel=0&amp;wmode=transparent}
\ \\
Dabei muss für die URL die src-Adresse von YouTube eingegeben werden, diese erhält man aus YouTube heraus, indem man auf \glqq Teilen\grqq\ klickt und dann auf \glqq embedded\grqq\ und daraus den src-Teil kopiert.
Der obige Player wurde mit folgendem Kommando erzeugt:
\begin{verbatim}
\MYoutubeVideo{Newtons Laws (2)}{400}{300}{https://www.youtube.com/embed/WzvhuQ5RWJE?rel=0&amp;wmode=transparent}
\end{verbatim}

\end{MXContent}

\begin{MXContent}{Views}{Views}{STD}
\MDeclareSiteUXID{BASE_SITE_VIEWS}
Es gibt die Möglichkeit Ansichten (Views) zu definieren. Das ist beispielsweise nützlich, um Kursinhalte weiter zu gliedern. So
ist eine View denkbar, die alle Videos enthält, oder eine View, die zum ausgewählten Thema Zusatzinformationen anzeigt.

Views können einfach mit dem folgenden Befehl definiert werden: \begin{verbatim}
\beginView{view-name}
<hier kommt der inhalt der view>
\endView{view-name}
\end{verbatim}
Views werden dabei automatisch vom Konverter erkannt und falls auf einer Seite vorhanden, oben angezeigt. Gemäß der Trennung von Inhalt und Darstellung
müssen/können die Icons, die für eine View angezeigt werden in css geändert werden.
\MTitle{View Icons}
Jede View bekommt einen eindeutigen css ID-Selektor zugewisen. So hat eine View mit dem Namen \texttt{videos} den ID-selektor \texttt{view-videos}.
Mit dem folgenden css code kann beispielsweise das Icon für die view-videos geändert werden.
\begin{verbatim}
#navbarTop #view-videos::before {
font-family: 'Glyphicons Halflings';
content: "\e105";
}
\end{verbatim}
Das setzt voraus, dass Glyphicons eingebunden wurden (per default der Fall). Die codes für den Content können hier gesehen werden
\MExtLink{https://glyphicons.bootstrapcheatsheets.com/}{bootstrap3 glyphicons cheatsheet}
\beginView{test}{}
Das ist die erste Ansicht
\endView{test}
\beginView{test2}{}
Das ist die zweite Ansicht
\endView{test2}

\MTitle{View Hilfetexte}
Entsprechend der Übersetzbarkeit der Inhalte werden die sprachspezifischen Texte im i18n ordner in den jeweiligen Sprachdateien editiert. Wenn für die
View aus obigem Beispiel ein Hilfetext definiert werden soll müss in die Datei \texttt{i18n/de.json} sowie in jede weitere unterstützte Sprachdatei die folgende
Zeile hinzugefügt werden:
\begin{verbatim}
"view-videos": "Video Ansicht mit zusätzlichem Kursmaterial",
\end{verbatim}


% Views müssen zuerst mit dem Befehl \begin{verbatim}\defineView{view-name}{viewoptionen}\end{verbatim} definiert werden. Dabei gibt \texttt{view-name}
% den Namen an, der der View zugewiesen wird. \texttt{optionen} hat die Syntax \texttt{option1=wert1;option2=wert2}. Erkannt werden
% die Optionen:

% \begin{MDTabular}{lll}
% Option & Beschreibung & default \\
% icon & der icon-name des glyphicons (siehe \MExtLink{https://getbootstrap.com/docs/3.3/components/}{bootstrap3 glyphicons}) & glyphicon-eye-open \\
% helptext-key & Eine id über die ein i18n Text abgerufen wird. Diese sind in \texttt{src/files/i18n/<lang>.json} gespeichert und müssen um den ausgewählten
% helptext-key für alle Sprachen ergänzt werden. & view-helptext
% \end{MDTabular}



\end{MXContent}


\end{document}
