\MSection{Aufgaben}

\MSubsection{Aufgabentypen}
\MLabel{SUBSECTION_EXERCISE_TYPES}

\begin{MXContent}{Checkbox}{Checkbox}{STD}
\MLabel{SUBSUBSECTION_CHECKBOX}

Eine Beispiel-Checkbox-Aufgabe inklusive Lösungsweg und richtigen Antworten.

\begin{MExercise}
Sind diese Ungleichungen richtig oder falsch?

\begin{MQuestionGroup}
\begin{tabular}{lll}
\MLCheckbox{0}{UG1} & \ \ &  $\frac12>1-\frac13$\\
\MLCheckbox{1}{UG2} & \ \ & $a^2\geq 2a b-b^2$ (wobei $a$ und $b$ unbekannte Zahlen sind)\\
\MLCheckbox{1}{UG3} & \ \ & $\frac12<\frac23<\frac34$\\
\MLCheckbox{0}{UG4} & \ \ & Angenommen $a<b$, dann ist immer auch $a^2<b^2$.
\end{tabular}
\end{MQuestionGroup}
\MGroupButton{Eingabe überprüfen}

\begin{MHint}{\iSolution}
Die erste Ungleichung vereinfacht sich zu $\frac12>\frac23$, was nach Multiplikation mit $6$ äquivalent ist zu $3>4$, dies ist eine falsche Aussage.\\
Die zweite Ungleichung kann man durch Übertragen aller Zahlenwerte auf die linke Seite
vereinfachen zu $a^2-2a b+b^2\geq 0$, dies ist eine wegen $a^2-2a b+b^2=(a-b)^2$ für alle Zahlen $a$ und $b$ wahre Aussage.\\
Multiplikation der dritten Ungleichungskette mit dem Hauptnenner $12$ ergibt die Kette $6<8<9$, die erfüllt ist.\\
Die letzte Aussage ist dagegen falsch, beispielsweise für $a=-1$ und $b=1$ ist $a^2=1$ nicht kleiner als $b^2=1$. Quadrieren von Termen ist keine Äquivalenzumformung.
\end{MHint}
\end{MExercise}

\end{MXContent}
