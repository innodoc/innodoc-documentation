% Damit die Moduldatei auch direkt mit pdflatex umgesetzt werden kann, müssen die Makros eingebunden werden
\input{mintmod.tex}

\MPragma{MathSkip} % Pragmas legen Konvertierungsoptionen nur für diese Datei fest. Diese Pragma fügt Abstände vor displaystyle-Matheumgebungen ein

\begin{document}

%\Mtableofcontents % erscheint nur bei PDF-Konvertierung, im HTML gibt es eine separate Seitennavigation

\MSection{Testmodul}

% Jede MSection besteht aus einer MSectionStart-Umgebung, dann MSubsections mit darin enthaltenen MXContent-Umgebungen.

\begin{MSectionStart}
% Jede Seite und jedes Frageelement haben eine UXID (unique extended ID), diese ist ein
% innerhalb des Kurses eindeutiger Textstring der die Seite unabhängig von Nummerierung und
% Anordnung intern identifiziert.
\MDeclareSiteUXID{BASE_EXAMPLE_TEST_MODULE}
Dieses Modul ist dafür da automatisierte Tests laufen zu lassen

\MModstartBox

\end{MSectionStart}

\MSubsection{Ein Unterabschnitt}
\MLabel{LABEL_BASE_SUBSECTION_FIRST}
\MDeclareSiteUXID{BASE_SUBSECTION_FIRST}
% Auch Unterabschnitte bekommen eine UXID

% Jegliche Inhalte müssen in MXContent oder MSectionStart-Umgebungen stehen.
% Jede solche Umgebung wird zu einer separaten HTML-Seite.
% Schema: \begin{MXContent}{Titel der Seite}{Kurztitel}{Typ} (Typ ist veraltet und sollte auf STD gesetzt werden)

\begin{MXContent}{Roulette Test}{Roulette Punkte Test}{STD}
\MDeclareSiteUXID{BASE_SITE_ONE}
\MLabel{LABEL_BASE_SITE_ONE} % LaTeX-Labels NIEMALS mit \label und \ref sondern mit den gekapselten Versionen deklarieren

In diesem Modul sollten insgesamt 8 Punkte zu erreichen sein. Es wurde eine Rouletteaufgabe eingebunden, die 2 Aufgaben mit jeweils 4 Punkten
beinhaltet.

\MTitle{Roulette Typ 2:}
Typ 2 Autorouletteaufgaben bieten immer nur eine Aufgabe der Aufgabenserie an und können so z.B. für Tests benutzt werden.
Die folgende Aufgabe wurde mit dem Befehl
\begin{verbatim}
\MAutoRouletteExercises {parabelaufgabe.py}{5}{Roulette2}{2}
\end{verbatim}
erzeugt.
\MAutoRouletteExercises{parabelaufgabe.py}{5}{Roulette2}{2}

\end{MXContent}

\end{document}
